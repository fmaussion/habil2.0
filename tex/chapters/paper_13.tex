\cleardoublepage

\renewcommand{\papertext}{Paper 13: Limited influence of climate change mitigation on short-term glacier mass loss}

\section*{\papertext}
\addcontentsline{toc}{section}{\protect\numberline{}\papertext}%
\label{paper_13}

\vspace{0.5cm}

\begin{singlespace}
\begin{hangparas}{1em}{1}

Marzeion, B., Kaser, G., \textbf{Maussion, F.} and Champollion, N.: Limited influence of climate change mitigation on short-term glacier mass loss, Nat. Clim. Chang., 8, \href{https://doi.org/10.1038/s41558-018-0093-1}{doi:10.1038/s41558-018-0093-1}, 2018.

\end{hangparas}
\end{singlespace}

\vspace{0.5cm}

Glaciers are a low-pass filter of atmospheric variability: their state at any time is the result of past climate
variability. In the present day context of rising global temperatures, many of them are what is
called “in disequilibrium” with the current climate: this means that even if greenhouse gas emissions would suddenly stop
and global warming would come to a halt, many glaciers would still loose mass for a period of time ranging from
years to centuries.

Although this disequilibrium was known and documented, very few studies attempted to quantify how much
glaciers would melt under such a hypothetical scenario. This melt is named “committed mass-loss”, i.e. the melt
that is still occurring as a result of our past emissions.

The study presented here quantified this committed mass loss to about 36\% of present day glacier volume (oustide
of Antarctica).
It also extended the range of analyzed
outcomes by running the Marzeion et al. 2012 model (the precursor of OGGM) under further global temperature
stabilization scenarios, including the 1.5 and 2K levels as envisioned in the Paris Agreement. We show that the difference
between the 1.5 and 2K scenarios is small during the 21st century, but remains significant for the long-term equilibrium
state of glaciers.

This study received considerable media attention, in particular because of the finding that at present day,
“one kilogram of glacier ice is lost every five hundred meters by car”. To reach this result, we could use
the almost linear relationship between global temperature and greenhouse gas emissions
to compute a direct relationship between CO\(_2\) emissions and amount of glacier melt.

The next phase of the GlacierMIP project (“GlacierMIP 3”) will complement and extend this study, by comparing
the equilibrium states of glaciers between the various GlacierMIP models.

I contributed to this paper during the conception and design phase, as well as during the
analysis phase. I was also involved with minor contributions to the writing.

\href{https://doi.org/10.1038/s41558-018-0093-1}{Link to the paper} (non open-access)


%\includepdf[pages=-,openright]{./papers/paper_13.pdf}

