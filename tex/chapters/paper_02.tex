\cleardoublepage

\renewcommand{\papertext}{Paper 02: How accurate are estimates of glacier ice thickness? Results from ITMIX, the Ice Thickness Models Intercomparison eXperiment}

\section*{\papertext}
\addcontentsline{toc}{section}{\protect\numberline{}\papertext}%
\label{paper_02}

\vspace{0.5cm}

\begin{singlespace}
\begin{hangparas}{1em}{1}
Farinotti, D., Brinkerhoff, D. J., Clarke, G. K. C., Fürst, J. J., Frey, H., Gantayat, P., Gillet-Chaulet, F., Girard, C., Huss, M., Leclercq, P. W., Linsbauer, A., Machguth, H., Martin, C., \textbf{Maussion, F.}, Morlighem, M., Mosbeux, C., Pandit, A., Portmann, A., Rabatel, A., Ramsankaran, R., Reerink, T. J., Sanchez, O., Stentoft, P. A., Singh Kumari, S., van Pelt, W. J. J., Anderson, B., Benham, T., Binder, D., Dowdeswell, J. A., Fischer, A., Helfricht, K., Kutuzov, S., Lavrentiev, I., McNabb, R., Gudmundsson, G. H., Li, H. and Andreassen, L. M.: How accurate are estimates of glacier ice thickness? Results from ITMIX, the Ice Thickness Models Intercomparison eXperiment, Cryosph., 11(2), 949--970, \href{https://doi.org/10.5194/tc-11-949-2017}{doi:10.5194/tc-11-949-2017}, 2017.
\end{hangparas}
\end{singlespace}

\vspace{0.5cm}

This paper is the result of an international working group of the International Association of
Cryospheric Sciences (IACS): the working group on \href{https://cryosphericsciences.org/activities/ice-thickness}{Glacier ice thickness estimation}  (2014--2019),
of which I was a member. The group activities led to several major publications, two of them are presented in
this thesis.

The first collaborative effort of the working group was the “Ice Thickness Models Intercomparison eXperiment (ITMIX)”,
phase 1. It was the first coordinated assessment of the individual performance of independent methods able to infer
glacier ice thickness from characteristics of the surface. A set of 17 different models were used to estimate the ice
thickness “blindly”, i.e. without using any observations for model calibration or tuning.

This publication is a milestone in the field of glacier ice thickness estimation. Cited more than 110 times to date (
Google Scholar), it laid the ground for a wealth of follow-up studies, leading several research groups (including mine)
to increase efforts in developing new methods to estimate the volume of glaciers. Indeed, we showed that the
disagreement between models themselves and between models and observations can be very large (up to several times the
observed ice thickness). Ensemble approaches may reduce model errors, but no model consistently outperformed the others.
A few models were favorably listed for their ability to robustly simulate many glaciers with a reasonable accuracy (e.g.
OGGM, Huss), or for their high accuracy on fewer glaciers (e.g. Brinkerhoff-v1),

My contribution to this paper was the participation in several meetings that helped to shape the experimental design,
and I participated with a model contribution. OGGM was ranked among the best models able to estimate ice thickness from
limited information (i.e. able to compute many glaciers). I also contributed to the analysis of the
results, and played a minor role in the writing of the paper.


\href{https://doi.org/10.5194/tc-11-949-2017}{Link to the paper} (open access).


\iflong \includepdf[pages=-,openright]{./papers/paper_02.pdf} \else \fi 

