\cleardoublepage

\renewcommand{\papertext}{Paper 12: Partitioning the Uncertainty of Ensemble Projections of Global Glacier Mass Change}

\section*{\papertext}
\addcontentsline{toc}{section}{\protect\numberline{}\papertext}%
\label{paper_12}

\vspace{0.5cm}

\begin{singlespace}
\begin{hangparas}{1em}{1}

Marzeion, B., Hock, R., Anderson, B., Bliss, A., Champollion, N., Fujita, K., Huss, M., Immerzeel, W., Kraaijenbrink, P., Malles, J., \textbf{Maussion, F.}, Radić, V., Rounce, D. R., Sakai, A., Shannon, S., Wal, R. and Zekollari, H.: Partitioning the Uncertainty of Ensemble Projections of Global Glacier Mass Change, Earth’s Futur., 8(7), \href{https://doi.org/10.1029/2019ef001470}{doi:10.1029/2019ef001470}, 2020.

\end{hangparas}
\end{singlespace}

\vspace{0.5cm}

This paper is the second publication of the Climate and Cryosphere (\href{https://www.climate-cryosphere.org}{CliC})
Glacier Model Intercomparison Project (\href{https://www.climate-cryosphere.org/mips/glaciermip}{GlacierMIP}).
The working group’s goal is to coordinate a model intercomparison of existing state-of-the-art
large-scale glacier models in order to identify current model deficiencies and data
or development needs.

The first contribution of GlacierMIP \citep{Hock2019} compared previously published glacier change projections
of six glacier models. This second activity was coordinated, by enforcing certain boundary conditions to all
models as well as the use of the same forcing datasets. After an international open call for participation,
the number of participating models rose to 11.

This study is therefore the most comprehensive consensus estimate of future glacier change to date.
It provides the basis figures for glacier related sea-level rise for the upcoming IPCC AR6 report.

Another major outcome of the paper is a much improved quantification of the sources of uncertainties
in future glacier change. “Uncertainty” here is defined as the variance of all possible future outcomes, and it is
divided into four main sources originating from the choice of (i) the Representative Concentration Pathway (RCP),
(ii) the General Circulation Model (GCM), (iii) the glacier model, and from (iv) natural climate variability.
While the uncertainty of the RCP scenario is dominating the total uncertainty by the end of the 21st century,
the dominant uncertainty in the first half of the century stems from differences between the glacier models.

These uncertainties have major implications for near- to medium term water resources and for adaptation planning:
reducing them will be a driving motivation for the OGGM project in the upcoming years.

I contributed to this study during the design phase (GlacierMIP meetings). I participated with one model contribution,
and minor contributions to the writing.

\href{https://doi.org/10.1029/2019ef001470}{Link to the paper} (open-access)

%\includepdf[pages=-,openright]{./papers/paper_12.pdf}

