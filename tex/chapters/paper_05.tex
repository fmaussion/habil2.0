\cleardoublepage

\renewcommand{\papertext}{Paper 05: Bias-corrected estimates of glacier thickness in the Columbia River Basin, Canada}

\section*{\papertext}
\addcontentsline{toc}{section}{\protect\numberline{}\papertext}%
\label{paper_05}

\vspace{0.5cm}

\begin{singlespace}
\begin{hangparas}{1em}{1}

Pelto, B. M., \textbf{Maussion, F.}, Menounos, B., Radić, V. and Zeuner, M.: Bias-corrected estimates of glacier thickness in the Columbia River Basin, Canada, J. Glaciol., 1--13, \href{https://doi.org/10.1017/jog.2020.75}{doi:10.1017/jog.2020.75}, 2020.

\end{hangparas}
\end{singlespace}

\vspace{0.5cm}

Ben Pelto approached me when he was working on his doctoral thesis about glaciers in the Columbia and Rocky Mountains,
Canada
(\href{https://unbc.arcabc.ca/islandora/object/unbc\%3A59097?solr\_nav\%5Bid\%5D=551a2e3c2ca38d6ce42a\&solr\_nav\%5Bpage\%5D=0\&solr\_nav\%5Boffset\%5D=1}{link to the thesis}).
His goal was to use OGGM to compute glacier thickness in the Columbia River Basin, using newly acquired ice thickness
and mass balance measurements.

While there exists a volume estimate for every glacier in the world (Paper 03), they can be subject to large
uncertainties at the single glacier or catchment scale. Using ground penetrating radar (GPR) observation campaigns, Ben
Pelto and colleagues found that the 2019 consensus ice thickness estimates were consistently too low for this catchment.

This study is a good example of the kind of collaboration that the OGGM project was designed for: using established
methods but adding value with new observational data, expert knowledge and an improved calibration scheme, we were able
to reduce model uncertainties and provide more accurate estimates of ice thickness in the region. The paper received
some local media attention after publication, which will surely benefit Ben Pelto’s career development.

I contributed to this paper by providing technical support with OGGM, as well as helping designing the calibration and
validation strategy, which involved using cross-validation to better estimate the out-of-sample accuracy of the model. I
also played a minor role in the writing of the paper.

\href{https://doi.org/10.1017/jog.2020.75}{Link to the paper} (open-access)


\iflong \includepdf[pages=-,openright]{./papers/paper_05.pdf} \else \fi 

