\cleardoublepage

\renewcommand{\papertext}{Paper 10: Initialization of a global glacier model based on present-day glacier geometry and past climate information: an ensemble approach}

\section*{\papertext}
\addcontentsline{toc}{section}{\protect\numberline{}\papertext}%
\label{paper_10}

\vspace{0.5cm}

\begin{singlespace}
\begin{hangparas}{1em}{1}

Eis, J., \textbf{Maussion, F.} and Marzeion, B.: Initialization of a global glacier model based on present-day glacier geometry and past climate information: an ensemble approach, Cryosph., 13(12), 3317--3335, \href{https://doi.org/10.5194/tc-13-3317-2019}{doi:10.5194/tc-13-3317-2019}, 2019.

\end{hangparas}
\end{singlespace}

\vspace{0.5cm}

What drove glacier retreat at the end of the Little Ice Age? How much did glaciers contribute to past sea-level rise?
What is the human influence on past glacier change? To answer all these questions, knowledge about past glacier states (area and volume)
is required. To date, only one of the available large-scale glacier models was able to provide elements of answer to
some of these questions \citep{Marzeion2015,Marzeion2014}.

The problem of reconstructing past glacier states from today’s geometry and past climate information is a difficult
inverse problem, often with non-unique solutions: many plausible past glacier states are compatible with today’s
glacier geometry and with past climate variability. The non-uniqueness of the problem is independent
of the quality of the data used to do the reconstructions. For these reason, in this paper we chose to address
the problem from a theoretical standpoint before trying to reconstruct past glacier states in a
real world context.

In this paper led by Julia Eis (first of her two first-author PhD thesis papers,
\href{https://media.suub.uni-bremen.de/handle/elib/4635}{link to the PDF}), we used OGGM
to design synthetic experiments in which all boundary conditions were perfectly known. We then
used OGGM again to attempt to reconstruct these past glacier states from limited information.

Although being theoretical (or, rather, thanks to its teheoretical nature),
this study brought important insights into
glacier-climate relationships and into a new “glacier reconstructability” metric that
we defined in the paper. The study led to a logical follow-up (Eis et al, accepted
for publication in Frontiers in Earth Science), where the method developed here
was extended and successfully applied to 517 glaciers for which we have available length
observations.

I contributed to the study design of this paper, provided support to Julia in using
OGGM, and I contributed to the writing of several parts of the final manuscript.

\href{https://doi.org/10.5194/tc-13-3317-2019}{Link to the paper} (open-access)


%\includepdf[pages=-,openright]{./papers/paper_10.pdf}

