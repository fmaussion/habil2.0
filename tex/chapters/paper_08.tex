\cleardoublepage

\renewcommand{\papertext}{Paper 08: Reanalysis of a 10-year record (2004--2013) of seasonal mass balances at Langenferner/Vedretta Lunga, Ortler Alps, Italy}

\section*{\papertext}
\addcontentsline{toc}{section}{\protect\numberline{}\papertext}%
\label{paper_08}

\vspace{0.5cm}

\begin{singlespace}
\begin{hangparas}{1em}{1}

Galos, S. P., Klug, C., \textbf{Maussion, F.}, Covi, F., Nicholson, L., Rieg, L., Gurgiser, W., Mölg, T. and Kaser, G.: Reanalysis of a 10-year record (2004--2013) of seasonal mass balances at Langenferner/Vedretta Lunga, Ortler Alps, Italy, Cryosph., 11(3), 1417--1439, \href{https://doi.org/10.5194/tc-11-1417-2017}{doi:10.5194/tc-11-1417-2017}, 2017.

\end{hangparas}
\end{singlespace}

\vspace{0.5cm}

Measurements of mass balance are the backbone of any glacier modelling effort.
“Traditional” measurements have been conducted since decades and form the longest available
time series of glacier mass balance. They are conducted by local authorities and scientists,
then curated and provided by the World Glacier Monitoring Service (\href{https://wgms.ch}{WGMS}).

OGGM (and many other glacier models) are relying heavily on these observations for model calibration
and validation, and potential
biases and errors will have consequences carried over to the model projections. It is therefore
very important to ensure the quality of the provided mass balances, and also revisit (“re-analyse”)
previous measurements under the light of modern knowledge and methods.

In this study led by Stephan Galos (at that time PhD student at the University of
Innsbruck), we use a combination of surface energy balance modelling and partial
observations to reconstruct and homogenize past mass balance time series. We
show that the reconstructed data can differ substantially from the original
estimates, and we discuss the importance of a careful planning of mass balance
measurement campaigns.

I contributed to this paper by providing guidance on the statistical analysis,
and I conducted the bootstrap uncertainty analysis, helping to quantify the
uncertainty involved in interpolating ablation stake data to the glacier scale.
I am also regularly involved in fieldwork and field courses for our
master students on this particular glacier (Langenferner).

\href{https://doi.org/10.5194/tc-11-1417-2017}{Link to the paper} (open-access)

%\includepdf[pages=-,openright]{./papers/paper_08.pdf}

