
My first “first-hand encounter” with the dramatic retreat of glaciers happened when I was 19, as I had to step down the
several ladders leading to the Mer de Glace in the region of Chamonix. These ladders allowed hikers to reach the glacier
from the Montenvers train station. In 2003, they were several dozens of meters long; they have been constantly
lengthened since then, to compensate for further glacier retreat \citep{Mourey2017}.

Beyond this sentimental aspect, glaciers have important societal impacts. They are contributing to sea-level rise, are
regulators of freshwater availability in many regions of the world, and they are sources of geohazards. Their documented
past fluctuations are a useful (but complex) recorder of past climates.

For all these reasons, mountain glaciers have been studied for centuries. More recently, the Intergovernmental Panel on
Climate Change (IPCC) published
a \href{https://www.ipcc.ch/srocc/}{Special Report on the Ocean and Cryosphere in a Changing Climate} (SROCC, 2019). It draws
a grim picture of the state of the cryosphere: I quote below parts of the summary for policy makers of Chapters 02
\citep{Hock2019a} and 04 \citep{Oppenheimer2019}.

\begin{quote}
\textit{Mass change of glaciers in all mountain regions (excluding the Canadian and Russian Arctic, Svalbard, Greenland and Antarctica) was very likely -490 \(\pm\) 100 kg m\(^{-2}\) yr\(^{-1}\) (-123 \(\pm\) 24 Gt yr\(^{-1}\)) in 2006--2015. Regionally averaged mass budgets were likely most negative (less than -850 kg m\(^{-2}\) y\(^{-1}\)) in the southern Andes, Caucasus and the European Alps/Pyrenees, and least negative in High Mountain Asia (-150 \(\pm\) 110 kg m\(^{-2}\) yr\(^{-1}\)) but variations within regions are strong.}
\end{quote}

\begin{quote}
\textit{Snow cover, glaciers and permafrost are projected to continue to decline in almost all regions throughout the 21st century (high confidence). (…) Projected glacier mass reductions between 2015--2100 are likely 22--44\% for RCP2.6 and 37--57\% for RCP8.5. In regions with mostly smaller glaciers and relatively little ice cover (…), glaciers will lose more than 80\% of their current mass by 2100 under RCP8.5 (medium confidence), and many glaciers will disappear regardless emission scenario (very high confidence).}
\end{quote}

\begin{quote}
\textit{Changes in snow and glaciers have changed the amount and seasonality of runoff in snow-dominated and glacier-fed river basins (very high confidence) with local impacts on water resources and agriculture (medium confidence). (…) In some glacier-fed rivers, summer and annual runoff have increased due to intensified glacier melt, but decreased where glacier melt water has lessened as glacier area shrinks. Decreases were observed especially in regions dominated by small glaciers, such as the European Alps (medium confidence).}
\end{quote}

\begin{quote}
\textit{River runoff in snow dominated and glacier-fed river basins will change further in amount and seasonality in response to projected snow cover and glacier decline (very high confidence) with negative impacts on agriculture, hydropower and water quality in some regions (medium confidence). (…) Projected trends in annual runoff vary substantially among regions, and can even be opposite in direction, but there is high confidence that in all regions average annual runoff from glaciers will have reached a peak that will be followed by declining runoff at the latest by the end of the 21st century.}
\end{quote}

\begin{quote}
\textit{Global mean sea-level (GMSL) is rising (virtually certain) and accelerating (high confidence). The sum of glacier and ice sheet contributions is now the dominant source of GMSL rise (very high confidence).}
\end{quote}

\begin{quote}
\textit{Future rise in GMSL caused by thermal expansion, melting of glaciers and ice sheets and land water storage changes, is strongly dependent on which Representative Concentration Pathway (RCP) emission scenario is followed. SLR at the end of the century is projected to be faster under all scenarios, including those compatible with achieving the long-term temperature goal set out in the Paris Agreement. GMSL will rise between 0.43 m (0.29--0.59 m, likely range; RCP2.6) and 0.84 m (0.61--1.10 m, likely range; RCP8.5) by 2100 (medium confidence) relative to 1986--2005.}
\end{quote}

The IPCC reports condense the results of decades of research, integrating a wealth of observational and modelling
studies at local, regional, and global scales. Such quantitative observations and projections of glacier change at large
scales have, for long, been possible only thanks to the use of ingenious upscaling methods to compensate for the limited
available data. By combining observations, modelling and rough estimates of glaciated area, pioneering studies helped to
quantify the contributions of glaciers and ice-caps to sea-level rise
\citep{Gregory1998,Raper2000,Wal2001,Braithwaite2002,Kaser2006a,Raper2006a,Meier2007,Cogley2009,Bahr2009}, providing
fundamental contributions to the 4th IPCC report on Climate Change (\href{https://www.ipcc.ch/report/ar4/wg1/}{AR4, 2007}).

However, the reputation of AR4 -- and, by extension, of the IPCC -- was damaged a few years later because of one
paragraph in the WG2 report (Ch. 10.6.2, “The Himalayan glaciers”), unrelated to the studies referenced here. This
paragraph wrongly stated that Himalayan glaciers were “likely to disappear by 2035”, despite being in contradiction with
the state of knowledge at that time. The statement led to negative press for the IPCC and undermined its credibility as
a whole. Despite these long-lasting negative consequences, it is probable that this “Himalaya-gate” also led to an
increase of efforts from the mountain glacier research community towards more quantitative assessments of the state of
glaciers and their change, as well as an improved communication about the role of glaciers for downstream hydrology
\citep{Kaser.etal_2010,Radic2010,Radic2011,Bolch2012,Immerzeel2012}.

Motivated by the discrepancies of the few estimates and projections of global mass loss of glaciers available in AR4,
the first globally complete inventory of glaciers (the \href{https://www.glims.org/RGI}{Randolph Glacier Inventory, RGI}) was
produced to fulfill the needs of the forthcoming Fifth Assessment Report of the Intergovernmental Panel on Climate
Change (\href{https://www.ipcc.ch/report/ar5/wg1/}{IPCC AR5}). The RGI (a collection of glacier outlines complemented with
attributes such as glacier type, hypsometry, etc.) fundamentally changed the way that regional and global glacier change
assessments would be conducted \citep{Pfeffer2014}. One of the first observational study to make use of this inventory
was the “Reconciled Estimate of Glacier Contributions to sea-level Rise: 2003 to 2009” by 
\cite{Gardner2013}, and it was followed by many others \citep{Brun2017,Dussaillant2019,Shean2020}. The RGI also allowed
scientists to compute new estimates of global glacier volume, better constraining their potential contribution to
sea-level rise \citep{Huss2012,Grinsted2013}. Finally, the RGI paved the way for the development of a wealth of glacier
evolution models able to simulate the evolution of each glacier individually, making the use of upscaling strategies
more accurate or even obsolete
\citep{Marzeion2012,Giesen2012,Anderson2012,Giesen2013,Hirabayashi2013,Radic2014,Huss2015,Kraaijenbrink2017,Sakai2017,Shannon2019,Rounce2020}. 
Such models can be used to reconstruct past contributions of glacier to 20th century sea-level rise
\citep{Marzeion2015}, the part of past glacier loss due to anthropogenic causes \citep{Marzeion2014}, or the impact of
glacier change on future seasonal runoff in glaciated basins \citep{Bliss2014,Huss2018,Rounce2020b}, to cite only a few
of their many applications.

The work presented in this thesis is fully embedded in this global context. All studies presented here have been written
between 2015 and 2020, in between IPCC’s AR5 and the upcoming AR6. Several of them are originating from large
international collaborations, such as the working group
on \href{https://cryosphericsciences.org/activities/ice-thickness/}{Glacier ice thickness estimation} from the International
Association of Cryospheric Sciences (IACS), or the Glacier Model Intercomparison
Project (\href{https://www.climate-cryosphere.org/mips/glaciermip}{GlacierMIP}) from the Climate and Cryosphere (CliC) core
project of the World Climate Research Programme (WCRP).

Started in 2014, the development of the \href{https://oggm.org}{Open Global Glacier Model} (OGGM) could build upon the
experience gained from the pioneering studies listed above, and many others that could not be referenced here.
Benefiting from constantly improving observational datasets, the model is in continuous development to adapt for new
boundary conditions, or new datasets for calibration and validation. At the time of writing, OGGM is now an established
modelling framework, representing the “state-of-the art” in large-scale glacier modelling.

\section*{Synopsis of the thesis}
\addcontentsline{toc}{section}{\protect\numberline{}Synopsis of the thesis}%


This habilitation thesis summarizes a few of the many contributions to the tremendous progress made in the field of
large-scale gaciology since 2015. Of course, it is limited to my own contributions and only offers my personal perspective,
as is required for such a document. The thesis is organized around the following thematic chapters and publications:
\begin{itemize}
\item {} 
In \textbf{Chapter 2}, I present the model description publication introducing OGGM, as well as several new developments
worth discussing here. OGGM is a modelling framework dealing with the entire glacier system, under several
simplifications required for large-scale applications. The following chapters therefore address three major aspects of
glacier system modelling.

\item {} 
In \textbf{Chapter 3}, I present four publications dealing with the topic of glacier ice thickness estimation using
numerical methods. Knowledge about glacier volume and bed topography is a prerequisite for the modelling of glacier
evolution.

\item {} 
In \textbf{Chapter 4}, I present four publications dedicated to glacier mass balance estimation, using in-situ and remote
observations combined with statistical and numerical methods. Surface mass balance is the most direct response of
glaciers to climate variations, and the observation of glacier mass balance provides invaluable calibration products
for glacier models.

\item {} 
In \textbf{Chapter 5}, I present four publications making use of the OGGM model to simulate past and future glacier change.
Two of them are focussing on the problem of reconstructing past glaciers length fluctuations, while the two others
deal with glacier change projections.

\item {} 
In \textbf{Chapter 6}, I introduce OGGM-Edu, an online educational platform about glaciers, built upon OGGM and for
high-school and university instructors.

\item {} 
In \textbf{Chapter 7}, I conclude this thesis and provide an outlook about planned future research based on OGGM.

\end{itemize}

Each thesis publication is preceded by a short introductions, meant to offer my personal
perspective on the content and genesis of the papers, as well as their relevance in the context
of this thesis.

I hope that you will enjoy this personal journey in the field of large-scale glacier modelling!