\addchap{Preface}


This habilitation thesis consists of 13 peer reviewed publications completed
between 2015 and 2020. It also contains two short, previously unpublished chapters
introducing what I believe to be my most important contributions to the field of
large-scale glaciology to date: (i) the \href{https://oggm.org}{Open Global Glacier Model (OGGM)},
an open-source glacier evolution model, and (ii) \href{https://edu.oggm.org}{OGGM-Edu},
a collaborative educational website about glaciers. All papers, web platforms
and software presented here were prepared while I was working at the University of Innsbruck.

This thesis addresses several aspects of the numerical modelling of glacier change,
at various spatial (glacier to global) and temporal (annual to centennial) scales. Thanks to
the projects I had the chance to contribute to, it also develops some aspects of observational
glaciology and meteorology. However, the main bulk of this work is focussing on the central
topic of this thesis: \textbf{the development of modern numerical methods to (i) estimate the ice
thickness of mountain glaciers and their volume, (ii) compute their mass balance and (iii)
simulate their evolution under climate change}.

A further important theme that is developed in this thesis is the topic of \textbf{open science}.
Because of my personal conviction that all scientific results should be openly available, I have
spent a lot of thought and energy on developing methods and workflows that enable a more open,
accessible, documented, and reproducible scientific practice. Many of these ideas are not from my
own invention, but are workflows that I borrowed and adapted from the open-source software
development community.

All the work presented here is the result of numerous collaborations, within the University of Innsbruck,
but also originating from a wider network of collaborators and from international working groups.
Five of these papers arose from my contributions
to the doctoral studies of Beatriz Recinos and Julia Eis (University of Bremen), Ben Pelto (University of
Northern British Columbia), Stephan Galos (University of Innsbruck) and the master thesis of
Tobias Zolles (University of Innsbruck). \\ [0.8cm]

\textit{This thesis is also available as a website:}
\href{https://fabienmaussion.info/habil2.0}{www.fabienmaussion.info/habil2.0} \\
\textit{The content is strictly the same - readers are free to choose which format suits them better!}
