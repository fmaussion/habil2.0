\cleardoublepage

\renewcommand{\papertext}{Paper 06: ENSO influence on surface energy and mass balance at Shallap Glacier, Cordillera Blanca, Peru}

\section*{\papertext}
\addcontentsline{toc}{section}{\protect\numberline{}\papertext}%
\label{paper_06}

\vspace{0.5cm}

\begin{singlespace}
\begin{hangparas}{1em}{1}

\textbf{Maussion, F.}, Gurgiser, W., Großhauser, M., Kaser, G. and Marzeion, B.: ENSO influence on surface energy and
mass balance at Shallap Glacier, Cordillera Blanca, Peru, Cryosph., 9(4), 1663--1683,
\href{https://doi.org/10.5194/tc-9-1663-2015}{doi:10.5194/tc-9-1663-2015}, 2015.

\end{hangparas}
\end{singlespace}

\vspace{0.5cm}

This study was conducted during my first PostDoc assignment at the University of Innsbruck, within the project
“Multidecadal to Centennial Climate Variability: Assessing the Conditions for the Glaciation of Tropical Mountains”.
Based on a 4-year long time series of distributed surface energy and mass balance (SEB/SMB, calculated using a
process-based model driven by observations), we developed a statistical model able to predict (downscale) monthly
SEB/SMB components from large-scale atmospheric reanalysis data. We then used these 34-yrs long reconstructions to study
the impact of ENSO on SEB and SMB fluxes at Shallap Glacier, Cordillera Blanca, Peru.

This paper also led to the publication of an open-source package written in
Python (\href{https://bitbucket.org/fmaussion/downglacier}{DownGlacier}). I could then apply some of the practices I
learned during the writing of this software to OGGM, which was only a vague project idea at that time. More importantly, this
study contributed to the growing literature and interest for the application of machine learning techniques in
glaciology. An important follow-up project (unrelated to me) is the \href{https://github.com/JordiBolibar/ALPGM}{ALPGM} model
\citep{Bolibar2020}, developed by Jordi Bolibar at the University of Grenoble.

My contribution to this paper was the development of the downscaling method, the generation of the figures and the
writing of the paper.

\href{https://doi.org/10.5194/tc-9-1663-2015}{Link to the paper} (open-access) \\
\href{https://bitbucket.org/fmaussion/downglacier}{Link to the open software “DownGlacier”}



%\includepdf[pages=-,openright]{./papers/paper_06.pdf}

