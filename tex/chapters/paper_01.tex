\cleardoublepage

\newcommand{\papertext}{Paper 01: Model description}

\section*{\papertext}
\addcontentsline{toc}{section}{\protect\numberline{}\papertext}%
\label{paper_01}

\vspace{0.5cm}

\begin{singlespace}
\begin{hangparas}{1em}{1}
\textbf{Maussion, F.}, Butenko, A., Champollion, N., Dusch, M., Eis, J., Fourteau, K., Gregor, P., Jarosch, A. H., Landmann, J., Oesterle, F., Recinos, B., Rothenpieler, T., Vlug, A., Wild, C. T. and Marzeion, B.: The Open Global Glacier Model (OGGM) v1.1, Geosci. Model Dev., 12(3), 909–931, \\
\href{https://doi.org/10.5194/gmd-12-909-2019}{doi:10.5194/gmd-12-909-2019}, 2019.
\end{hangparas}
\end{singlespace}

\vspace{0.5cm}

This paper is a milestone in the development of OGGM, marking its first stable release (v1.0), and officially marking
its entry in the suite of available glacier models. OGGM has been used in two studies prior to this
publication (Paper 02, Paper 11): this was made possible by opening the code and
writing the model documentation long before submitting this paper.

Written for the journal “\href{https://gmd.copernicus.org}{Geoscientific Model Development}”, this paper is targeting
prospective model users. It provides the necessary information about the model structure, its physical assumptions, and
examples of applications. It also discusses some of
its \href{https://docs.oggm.org/en/stable/pitfalls.html}{pitfalls and limitations}.

My contribution to this study was the writing of the paper, the creation of all figures, the lead of the OGGM project
and the development of the main bulk of the OGGM codebase. All co-authors of this publication participated actively in
the OGGM development, and OGGM would not be where it is now without their contribution.

\href{https://doi.org/10.5194/gmd-12-909-2019}{Link to the paper} (open access).

\iflong \includepdf[pages=-,openright]{./papers/paper_01.pdf} \else \fi 
