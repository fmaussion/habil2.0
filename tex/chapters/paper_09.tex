\cleardoublepage

\renewcommand{\papertext}{Paper 09: Global glacier mass changes and their contributions to sea-level rise from 1961 to 2016}

\section*{\papertext}
\addcontentsline{toc}{section}{\protect\numberline{}\papertext}%
\label{paper_09}

\vspace{0.5cm}

\begin{singlespace}
\begin{hangparas}{1em}{1}

Zemp, M., Huss, M., Thibert, E., Eckert, N., McNabb, R., Huber, J., Barandun, M., Machguth, H., Nussbaumer, S. U., Gärtner-Roer, I., Thomson, L., Paul, F., \textbf{Maussion, F.}, Kutuzov, S. and Cogley, J. G.: Global glacier mass changes and their contributions to sea-level rise from 1961 to 2016, Nature, 568(7752), 382--386, \href{https://doi.org/10.1038/s41586-019-1071-0}{doi:10.1038/s41586-019-1071-0}, 2019.

\end{hangparas}
\end{singlespace}

\vspace{0.5cm}

The reconstruction of past global glacier change is severely hindered by the lack of observations at large-scales.
Direct glaciological measurements of mass balances (such as presented in Paper 08)
have been performed on a few hundreds of glaciers worldwide (about 0.1\% of all glaciers on Earth). More recently,
satellite derived geodetic estimated have strongly increased the spatial coverage of mass balance estimates,
but they are available for the past two decades at most (and, at the time of writing, not yet available globally).

This comprehensive meta-analysis of all data available at the WGMS is a considerable effort to reconstruct
global glacier change back to 1961. Using a well established (and sometimes forgotten) multivariate statistical
interpolation method \citep{Lliboutry1974} and robust uncertainty estimates, this reconstruction
allowed to revisit
previous global estimates and confirm the crucial contribution of past and present glacier change to sea-level rise.
Amongst the many contributions of this paper to the glacier literature, on a personal and scientific level I
particularly praise the careful handling of the (large) uncertainties, which are very openly communicated.

My involvement in this effort started during a three days workshop, during which all authors
of the final study discussed and debated the various ways to merge and consolidate these highly heterogeneous
data sources. I then performed a semi-automated clustering analysis to define the glacier regions
on which the Lliboutry method could be applied, and had a minor contribution to the writing of the
paper.

\href{https://doi.org/10.1038/s41586-019-1071-0}{Link to the paper} (non open-access)

\iflong \includepdf[pages=-,openright]{./papers/paper_09.pdf} \else \fi 

